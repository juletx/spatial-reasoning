This chapter describes the Visual Spatial Reasoning (VSR) \cite{liu2022visual} dataset and the splits that are used for evaluation. We also explain the experiments we performed and the results we obtained in VSR.

\section{Dataset}

The objective of VSR is to \textbf{test spatial grounding} capabilities by covering \textbf{65 spatial relations} over natural images from COCO. Given an image and a caption which describes a spatial relation between two of the objects, the model has to infer if the relation is true or false.

A \textbf{contrastive caption generation} approach was used in VSR to avoid choosing too many trivial relations. First, a pair of images that contain the same two concepts are selected from COCO. Second, an annotator had to choose a spatial relation that made the caption template correct for one image but incorrect for the other. Finally, every item is reviewed by at least two additional human annotators. If the agreement between annotators is not high enough, the data point is excluded.

To get a more high-level understanding of the relations, they are grouped in \textbf{meta categories} \cite{marchi2021cross}: Adjacency, Directional, Orientation, Projective, Proximity, Topological and Unallocated (see \cref{tab:spatial_relations}). We show some examples to understand the differences between relation categories in Figures \ref{fig:vsr-examples} and \ref{fig:vsr-examples-2}.

\begin{table}[ht]
    \centering
    \begin{adjustbox}{max width=\textwidth}
    \begin{tabular}{l|l}
    \toprule
        \rowcolor{DarkGray}
    Category & Spatial Relations \\
    \midrule
    Adjacency   & \makecell[l]{Adjacent to, alongside, at the side of, at the right side of, at the left side of, attached to, at the back of,\\ ahead of, against, at the edge of} \\
    \rowcolor{Gray}
 Directional & \makecell[l]{Off, past, toward, down, deep down$^\ast$, up$^\ast$, away from, along, around, from$^\ast$, into, to$^\ast$, across, across from, \\through$^\ast$, down from }\\
    Orientation & Facing, facing away from, parallel to, perpendicular to\\
    \rowcolor{Gray}
    Projective & On top of, beneath, beside, behind, left of, right of, under, in front of, below, above, over, in the middle of\\
    Proximity & By, close to, near, far from, far away from \\
        \rowcolor{Gray}
    Topological & \makecell[l]{Connected to, detached from, has as a part, part of, contains, within, at, on, in, with, surrounding, among, \\ consists of, out of, between, inside, outside, touching}\\
    Unallocated & Beyond, next to, opposite to, after$^\ast$, among, enclosed by \\
\bottomrule
    \end{tabular}
    \end{adjustbox}
    \caption{The available 71 spatial relations. 65 of them appear in the final dataset. Relations with $\ast$ are not used.}
    % unused: 'up', 'through', 'deep down', 'from', 'to', 'after'
    \label{tab:spatial_relations}
\end{table}

In \cref{fig:vsr-examples} we show examples of Adjacency, Projective and Topological meta categories. \textbf{Adjacency} examples involve identifying what is ahead of the cow and which is the edge of the table. The \textbf{Projective} images are paired with the same caption, but have different labels. \textbf{Topological} examples require understanding what being inside and touching are.

\begin{figure}[ht]
\centering
    \begin{minipage}[t]{.30\textwidth}
        \begin{subfigure}[t]{\textwidth}
        \centering
        \includegraphics[height=3cm]{000000080336.jpg}
        \caption{Caption: \textit{The person is ahead of the cow.} Label: \texttt{True}.}
        \label{fig:person_cow}
        \end{subfigure}\\
        \begin{subfigure}[t]{\textwidth}
        \centering
        \includegraphics[height=3cm]{000000261511.jpg}
        \caption{Caption: \textit{The pizza is at the edge of the dining table.} Label: \texttt{True}.}
        \label{fig:pizza_table}
        \end{subfigure}%
        \caption*{\textit{Adjacency}}
    \end{minipage}
    \hfill
    \begin{minipage}[t]{.30\textwidth}
        \begin{subfigure}[t]{\textwidth}
        \centering
        \includegraphics[height=3cm]{000000119360.jpg}
        \caption{Caption: \textit{The cat is behind the laptop.} Label: \texttt{True}.}
        \end{subfigure}\\
        \begin{subfigure}[t]{\textwidth}
        \centering
        \includegraphics[height=3cm]{000000310958.jpg}
        \caption{Caption: \textit{The cat is behind the laptop.} Label: \texttt{False}.}
        \end{subfigure}% 
        \caption*{\textit{Projective}}
    \end{minipage}
    \hfill
    \begin{minipage}[t]{.30\textwidth}
        \begin{subfigure}[t]{\textwidth}
        \centering
        \includegraphics[height=3cm]{000000292365.jpg}
        \caption{Caption: \textit{The cat is inside the toilet.} Label: \texttt{False}.}
        \label{fig:cat_toilet_true}
        \end{subfigure}\\
        \begin{subfigure}[t]{\textwidth}
        \centering
        \includegraphics[height=3cm]{000000092020.jpg}
        \caption{Caption: \textit{The person is touching the hair drier.} Label: \texttt{True}.}
        \label{fig:cat_toilet}
        \end{subfigure}%
        \caption*{\textit{Topological}}
    \end{minipage}%
    \caption{Examples from the VSR dataset for the relation meta categories \textit{Adjacency}, \textit{Projective} and \textit{Topological} from left to right.}
    \label{fig:vsr-examples}
\end{figure}

In \cref{fig:vsr-examples-2} Adjacency, Projective and Orientation meta categories. The first \textbf{Adjacency} example is tricky, it requires knowing which is the right side of the bench. The second one is even more difficult because the cow both the cow appears in the car’s side mirror. \textbf{Projective} examples involve knowing where is the front of the person and below the cat. \textbf{Orientation} examples require understanding the orientations of the hair drier and the fire hydrant.

\begin{figure}[ht]
\centering
    \begin{minipage}[t]{.30\textwidth}
        \begin{subfigure}[t]{\textwidth}
        \centering
        \includegraphics[height=3cm]{000000259555.jpg}
        \caption{Caption: \textit{The potted plant is at the right side of the bench.} Label: \texttt{True}.}
        \end{subfigure}\\
        \begin{subfigure}[t]{\textwidth}
        \centering
        \includegraphics[height=3cm]{000000512796.jpg}
        \caption{Caption: \textit{The cow is at the back of the car.} Label: \texttt{True}.}
        \end{subfigure}%
        \caption*{\textit{Adjacency}}
    \end{minipage}
    \hfill
    \begin{minipage}[t]{.30\textwidth}
        \begin{subfigure}[t]{\textwidth}
        \centering
        \includegraphics[height=3cm]{000000434410.jpg}
        \caption{Caption: \textit{The bench is in front of the person.} Label: \texttt{True}.}
        \end{subfigure}\\
        \begin{subfigure}[t]{\textwidth}
        \centering
        \includegraphics[height=3cm]{000000420344.jpg}
        \caption{Caption: \textit{The keyboard is below the cat.} Label: \texttt{True}.}
        \end{subfigure}%
        \caption*{\textit{Projective}}
    \end{minipage}
    \hfill
    \begin{minipage}[t]{.30\textwidth}
        \begin{subfigure}[t]{\textwidth}
        \centering
        \includegraphics[height=3cm]{000000134738.jpg}
        \caption{Caption: \textit{The hair drier is facing away from the person.} Label: \texttt{False}.}
        \end{subfigure}\\
        \begin{subfigure}[t]{\textwidth}
        \centering
        \includegraphics[height=3cm]{000000147270.jpg}
        \caption{Caption: \textit{The fire hydrant is facing away from the person.} Label: \texttt{True}.}
        \end{subfigure}% 
        \caption*{\textit{Orientation}}
    \end{minipage}%
    \caption{Examples from the VSR dataset for the relation meta categories \textit{Adjacency}, \textit{Projective} and \textit{Orientation} from left to right.}
    \label{fig:vsr-examples-2}
\end{figure}

\section{Dataset Splits}\label{sec:vsr_splits}

The VSR dataset has two types of splits \cite{liu2022visual}, random and zero-shot. The statistics of the two splits are shown in \cref{tab:data_splits}.

\begin{table}[ht]
\small
\centering
\begin{tabular}{lllll}
\toprule
 split & train & dev & test & total   \\
\midrule
\textit{random} & 7,083 & 1,012 & 2,024 & 10,119 \\
\textit{zero-shot} & 5,440 & 259 & 731  & 6,430\\
\bottomrule
\end{tabular}
\caption{Data statistics of the \textit{random} and \textit{zero-shot} splits. }
\label{tab:data_splits}
\end{table}

\paragraph{Random split.}
The dataset is split randomly into train/dev/test with the ratio of 70\%/10\%/20\%. All the validated data points are used in this split.

\paragraph{Zero-shot split.}
It is a concept zero-shot split where train/dev/test have no overlapping concepts. That is, each concept can only appear in one of the sets.
This is done by randomly grouping concepts into three sets with the ratio of 50\%/20\%/30\%.
This is a more challenging setup because the model has to learn concepts and relations in a compositional way instead of remembering the co-occurrence of the two.
Moreover, having less training data is a disadvantage for the models, since not all the data can be used in this setting.

\section{Metrics}

The only metric used for evaluation is \textbf{accuracy}. Due to the fluctuations we observe, authors recommend always reporting the average performance of three runs to make sure the conclusion is
reliable \cite{liu2022visual}. In general, models have larger standard deviations on the zero-shot split, probably because the zero-shot dev/test sets are smaller.

\section{Models} \label{sec:vsr_models}

We introduce baseline models and our models.

\paragraph{Baselines.} VSR authors \cite{liu2022visual} test three popular VLMs: VisualBERT \cite{li2019visualbert}, 
LXMERT \cite{tan2020lxmert}, and
ViLT \cite{kim2021vilt}. All three models are stacked Transformers \cite{vaswani2017attention} that take image and text pairs as input. The difference mainly lies in how or whether they encode position information of objects. Checkpoints are saved every 100 iterations and the best checkpoint on the dev set is used for testing. All models are run three times using three random seeds.

\paragraph{Ours.} We first test the same baseline models. We also evaluate a ViLT model that has only been finetuned on NLVR2. We evaluate BLIP \cite{li2022blip} trained on VSR and NLVR2.

\section{Results} \label{sec:vsr_results}

\subsection{Compared To Humans}

\paragraph{Baseline}

See \cref{tab:vsr_results_base}. The gap between dev and tests becomes much greater on zero-shot split likely due to the smaller size of both dev and test sets.

\begin{table}[ht]
\centering
%\setlength{\tabcolsep}{2.8pt}
\small
\begin{tabular}{lcccccc}
\toprule
& \multicolumn{2}{c}{random split} &  & \multicolumn{2}{c}{zero-shot split} \\
\cmidrule(l){2-3} 	\cmidrule(l){4-6}
model$\downarrow$ & dev & test & & dev & test  \\
\midrule
human & \multicolumn{5}{c}{95.4}   \\
\midrule
VisualBERT & 59.2$_{\pm0.9}$ & 57.4$_{\pm0.9}$ & & 57.4$_{\pm2.2}$  & 54.0$_{\pm1.3}$  \\ 
LXMERT & \textbf{73.8}$_{\pm1.2}$  & \textbf{72.5}$_{\pm1.4}$ & & \textbf{69.2}$_{\pm1.0}$  & \textbf{63.2}$_{\pm1.7}$  \\ 
ViLT & 71.9$_{\pm1.3}$  & 71.0$_{\pm0.7}$  & & 66.7$_{\pm1.7}$  & 62.4$_{\pm1.5}$ \\ 
\bottomrule
\end{tabular}
\caption{Model performance on VSR. Results of both random and zero-shot splits, both validation and tests are listed.}
\label{tab:vsr_results_base}
\end{table}

\paragraph{Ours}

\subsection{Results By Relation}

\paragraph{Baseline}

See \cref{fig:performance_by_rel_base}

\begin{figure*}
    \centering
\begin{subfigure}[b]{\linewidth}
    \centering
    \includegraphics[width=\linewidth]{images/visual-spatial-reasoning/performance_by_relation_random_split_v2.png}
    \vspace{-1cm}
    \caption{random split}
\end{subfigure}
\begin{subfigure}[b]{\linewidth}
    \centering
    \includegraphics[width=\linewidth]{images/visual-spatial-reasoning/performance_by_relation_zeroshot_split_v2.png}
    \vspace{-1cm}
    \caption{zero-shot split}
\end{subfigure}
\caption{Performance by relation on the random (upper) and zero-shot (lower) split test sets. Relation order sorted by frequency (high to low from left to right). Only relations with more than 15 and 5 occurrences on the random and zero-shot tests respectively are shown. }
    \label{fig:performance_by_rel_base}
\end{figure*}

\paragraph{Ours}

See \cref{fig:performance_by_rel}

\begin{figure}[ht]
    \centering
\begin{subfigure}[b]{\linewidth}
    \centering
    \includegraphics[width=\linewidth]{images/visual-spatial-reasoning/performance_rel_random.png}
    \vspace{-1cm}
    \caption{random split}
\end{subfigure}
\begin{subfigure}[b]{\linewidth}
    \centering
    \includegraphics[width=\linewidth]{images/visual-spatial-reasoning/performance_rel_zeroshot.png}
    \vspace{-1cm}
    \caption{zero-shot split}
\end{subfigure}
\caption{Performance by relation on the random (upper) and zero-shot (lower) split test sets. Relation order sorted by frequency (high to low from left to right). Only relations with more than 15 and 5 occurrences on the random and zero-shot tests respectively are shown. }
    \label{fig:performance_by_rel}
\end{figure}

See \cref{tab:results-by-relation-random} and \cref{tab:results-by-relation-zeroshot}

\begin{table}[ht]
\centering
\begin{tabular}{lrrrrrr}
\toprule
relation &  number &  VisualBERT &  LXMERT &  ViLT &  ViLT NLVR2 &  BLIP NLVR2 \\
\midrule
all                  &    2024 &        55.1 &    73.9 &  71.2 &        59.1 &        60.1 \\
\midrule
touching             &     236 &        55.9 &    76.7 &  73.7 &        64.0 &        62.3 \\
behind               &     136 &        44.9 &    75.0 &  70.6 &        52.9 &        58.1 \\
on                   &     128 &        64.8 &    82.0 &  86.7 &        71.9 &        70.3 \\
in front of          &     116 &        54.3 &    70.7 &  63.8 &        58.6 &        65.5 \\
under                &     112 &        62.5 &    85.7 &  83.9 &        62.5 &        66.1 \\
on top of            &      87 &        50.6 &    79.3 &  79.3 &        72.4 &        67.8 \\
at the right side of &      85 &        51.8 &    76.5 &  57.6 &        63.5 &        50.6 \\
at the left side of  &      80 &        48.8 &    73.8 &  61.3 &        50.0 &        56.2 \\
beneath              &      80 &        63.7 &    80.0 &  77.5 &        58.8 &        56.2 \\
above                &      72 &        59.7 &    76.4 &  72.2 &        55.6 &        62.5 \\
contains             &      57 &        56.1 &    80.7 &  86.0 &        56.1 &        50.9 \\
in                   &      51 &        68.6 &    82.4 &  84.3 &        60.8 &        58.8 \\
facing               &      50 &        50.0 &    64.0 &  62.0 &        60.0 &        62.0 \\
far away from        &      49 &        51.0 &    77.6 &  75.5 &        40.8 &        42.9 \\
inside               &      49 &        59.2 &    77.6 &  79.6 &        57.1 &        55.1 \\
below                &      42 &        59.5 &    66.7 &  66.7 &        47.6 &        52.4 \\
next to              &      41 &        56.1 &    68.3 &  75.6 &        53.7 &        65.9 \\
at the edge of       &      40 &        42.5 &    47.5 &  60.0 &        50.0 &        62.5 \\
left of              &      39 &        56.4 &    76.9 &  59.0 &        59.0 &        56.4 \\
beside               &      34 &        44.1 &    73.5 &  64.7 &        79.4 &        67.6 \\
facing away from     &      32 &        56.2 &    53.1 &  46.9 &        56.2 &        50.0 \\
away from            &      31 &        61.3 &    71.0 &  74.2 &        41.9 &        64.5 \\
right of             &      24 &        50.0 &    87.5 &  58.3 &        58.3 &        54.2 \\
far from             &      23 &        47.8 &    87.0 &  87.0 &        43.5 &        56.5 \\
close to             &      21 &        57.1 &    71.4 &  71.4 &        71.4 &        57.1 \\
part of              &      21 &        42.9 &    76.2 &  76.2 &        42.9 &        42.9 \\
near                 &      21 &        52.4 &    57.1 &  71.4 &        76.2 &        66.7 \\
parallel to          &      19 &        31.6 &    36.8 &  57.9 &        52.6 &        47.4 \\
at the back of       &      19 &        57.9 &    73.7 &  63.2 &        52.6 &        63.2 \\
across from          &      18 &        66.7 &    72.2 &  66.7 &        44.4 &        44.4 \\
over                 &      16 &        50.0 &    75.0 &  93.8 &        81.2 &        56.2 \\
in the middle of     &      15 &        46.7 &    60.0 &  33.3 &        33.3 &        53.3 \\
off                  &      15 &        33.3 &    40.0 &  40.0 &        26.7 &        46.7 \\
\bottomrule
\end{tabular}
\caption{Number and performance by relation on the random split test. Only relations with more than 15 occurrences are shown.}
\label{tab:results-by-relation-random}
\end{table}

\begin{table}[ht]
\centering
\begin{tabular}{lrrrrrr}
\toprule
relation &  number &  VisualBERT &  LXMERT &  ViLT &  ViLT NLVR2 &  BLIP NLVR2 \\
\midrule
all                  &     731 &        50.8 &    65.5 &  61.6 &        52.8 &        53.9 \\
\midrule
in front of          &      76 &        46.1 &    64.5 &  53.9 &        50.0 &        52.6 \\
behind               &      71 &        49.3 &    78.9 &  69.0 &        50.7 &        49.3 \\
far away from        &      57 &        57.9 &    59.6 &  59.6 &        40.4 &        36.8 \\
at the left side of  &      32 &        59.4 &    71.9 &  50.0 &        59.4 &        71.9 \\
next to              &      32 &        40.6 &    62.5 &  62.5 &        81.2 &        65.6 \\
contains             &      29 &        48.3 &    86.2 &  75.9 &        48.3 &        55.2 \\
touching             &      27 &        55.6 &    48.1 &  77.8 &        55.6 &        74.1 \\
at the right side of &      25 &        44.0 &    48.0 &  68.0 &        52.0 &        72.0 \\
on                   &      23 &        52.2 &    87.0 &  78.3 &        73.9 &        82.6 \\
above                &      22 &        54.5 &    59.1 &  59.1 &        54.5 &        45.5 \\
left of              &      22 &        59.1 &    86.4 &  54.5 &        54.5 &        54.5 \\
on top of            &      20 &        40.0 &    80.0 &  90.0 &        85.0 &        85.0 \\
under                &      20 &        45.0 &    60.0 &  45.0 &        45.0 &        40.0 \\
below                &      18 &        66.7 &    61.1 &  61.1 &        61.1 &        66.7 \\
in                   &      16 &        37.5 &    87.5 &  87.5 &        62.5 &        75.0 \\
parallel to          &      14 &        35.7 &    42.9 &  50.0 &        64.3 &        42.9 \\
far from             &      14 &        57.1 &    71.4 &  71.4 &        28.6 &        42.9 \\
facing               &      14 &        50.0 &    42.9 &  78.6 &        71.4 &        57.1 \\
at the back of       &      14 &        71.4 &    64.3 &  50.0 &        35.7 &        35.7 \\
across from          &      14 &        42.9 &    57.1 &  57.1 &        28.6 &        28.6 \\
ahead of             &      13 &        30.8 &    53.8 &  38.5 &        53.8 &        61.5 \\
away from            &      12 &        50.0 &    41.7 &  50.0 &        33.3 &        41.7 \\
beside               &      12 &        41.7 &    41.7 &  66.7 &        41.7 &        25.0 \\
adjacent to          &      12 &        83.3 &    66.7 &  33.3 &        58.3 &        58.3 \\
right of             &      12 &        58.3 &    75.0 &  58.3 &        50.0 &        33.3 \\
beneath              &      11 &        54.5 &    63.6 &  54.5 &        54.5 &        45.5 \\
facing away from     &      10 &        60.0 &    70.0 &  60.0 &        60.0 &        50.0 \\
inside               &       8 &        50.0 &    62.5 &  75.0 &        37.5 &        75.0 \\
close to             &       8 &        62.5 &    50.0 &  37.5 &        50.0 &        37.5 \\
beyond               &       6 &        33.3 &    66.7 &  66.7 &        33.3 &        66.7 \\
alongside            &       6 &        33.3 &    66.7 &  66.7 &        50.0 &        50.0 \\
off                  &       6 &        66.7 &    50.0 &  50.0 &        16.7 &        33.3 \\
surrounding          &       5 &        40.0 &   100.0 &  60.0 &        40.0 &        80.0 \\
\bottomrule
\end{tabular}
\caption{Number and performance by relation on the zero-shot split test. Only relations with more than 5 occurrences are shown.}
\label{tab:results-by-relation-zeroshot}
\end{table}

\subsection{Results By Relation Meta Category}

\paragraph{Baseline}

See \cref{fig:performance_by_meta_cat_base}

\begin{figure*}
    \centering
\begin{subfigure}[b]{0.49\linewidth}
    \centering
    \includegraphics[width=\linewidth]{images/visual-spatial-reasoning/performance_by_meta_cat_random_split_v2.png}
    \caption{random split}
\end{subfigure}
\begin{subfigure}[b]{0.49\linewidth}
    \centering
    \includegraphics[width=\linewidth]{images/visual-spatial-reasoning/performance_by_meta_cat_zeroshot_split_v2.png}
    \caption{zero-shot split}
\end{subfigure}
\caption{Performance by meta categories of relations, on the random (left) and zero-shot (right) split test sets. For legend information, see \Cref{fig:performance_by_rel_base}.}
    \label{fig:performance_by_meta_cat_base}
\end{figure*}

\paragraph{Ours}

See \cref{fig:performance_by_meta_cat}

\begin{figure}[ht]
    \centering
    \begin{subfigure}[b]{0.49\linewidth}
    \centering
    \includegraphics[width=\linewidth]{images/visual-spatial-reasoning/performance_rel_meta_cat_random.png}
    \caption{random split}
     \end{subfigure}
     \begin{subfigure}[b]{0.49\linewidth}
         \centering
    \includegraphics[width=\linewidth]{images/visual-spatial-reasoning/performance_rel_meta_cat_zeroshot.png}
         \caption{zero-shot split}
     \end{subfigure}
\caption{Performance by meta categories of relations, on the random (left) and zero-shot (right) split test sets. For legend information, see \cref{fig:performance_by_rel}.}
    \label{fig:performance_by_meta_cat}
\end{figure}

See \cref{tab:results-by-relation-meta-category-random} and \cref{tab:results-by-relation-meta-category-zeroshot}

\begin{table}[ht]
\centering
\begin{tabular}{lrrrrrr}
\toprule
category &  number &  VisualBERT &  LXMERT &  ViLT &  ViLT NLVR2 &  BLIP NLVR2 \\
\midrule
All         &    2024 &        55.1 &    73.9 &  71.2 &        59.1 &        60.1 \\
\midrule
Adjacency   &     284 &        51.4 &    71.1 &  63.0 &        56.7 &        60.2 \\
Directional &      90 &        56.7 &    68.9 &  55.6 &        47.8 &        54.4 \\
Orientation &     112 &        50.9 &    55.4 &  54.5 &        55.4 &        56.2 \\
Proximity   &     123 &        52.0 &    73.2 &  74.8 &        53.7 &        52.8 \\
Projective  &     773 &        54.5 &    76.7 &  71.7 &        59.8 &        61.4 \\
Topological &     591 &        59.2 &    76.8 &  79.2 &        63.5 &        61.4 \\
Unallocated &      51 &        52.9 &    64.7 &  74.5 &        54.9 &        60.8 \\
\bottomrule
\end{tabular}
\caption{Number and performance by relation meta category on the random split test.}
\label{tab:results-by-relation-meta-category-random}
\end{table}

\begin{table}[ht]
\centering
\begin{tabular}{lrrrrrr}
\toprule
category &  number &  VisualBERT &  LXMERT &  ViLT &  ViLT NLVR2 &  BLIP NLVR2 \\
\midrule
All         &     731 &        50.8 &    65.5 &  61.6 &        52.8 &        53.9 \\
\midrule
Adjacency   &     114 &        55.3 &    62.3 &  52.6 &        55.3 &        61.4 \\
Directional &      40 &        50.0 &    52.5 &  52.5 &        25.0 &        40.0 \\
Orientation &      42 &        47.6 &    47.6 &  59.5 &        64.3 &        47.6 \\
Proximity   &      83 &        59.0 &    59.0 &  57.8 &        41.0 &        37.3 \\
Projective  &     286 &        50.0 &    69.6 &  61.2 &        53.5 &        51.0 \\
Topological &     124 &        48.4 &    74.2 &  76.6 &        55.6 &        67.7 \\
Unallocated &      42 &        38.1 &    64.3 &  61.9 &        71.4 &        64.3 \\
\bottomrule
\end{tabular}
\caption{Number and performance by relation meta category on the zero-shot split test.}
\label{tab:results-by-relation-meta-category-zeroshot}
\end{table}
