This chapter is an introduction of the master thesis and includes background, objectives and research questions.

\section{Background}

Neural Language Models (LM) have shown \textbf{impressive capabilities} on many Natural Language Processing (NLP) tasks \cite{wang2019superglue, brown2020language, chowdhery2022palm}. LMs are pretrained on large corpora in order for them to learn universal language representations, which are beneficial for downstream NLP tasks and can avoid training a new model from scratch. The pretrained models are then fine-tuned in specific downstream tasks, using annotated data that is orders of magnitude smaller than the text used in the pretraining phase. Following this transfer learning methodology, researchers have extended the state of the art on a wide array of tasks as measured by leaderboards on specific benchmarks for English \cite{bommasani2021opportunities, wang2019superglue}.

Despite the impressive results of LMs for different language-related tasks, many authors criticize them for their \textbf{lack of meaning} \cite{bender2020climbing, bender2021dangers}. In their opinion, language models trained exclusively on linguistic form (i.e. words) are unable to learn meaning. Those authors suggest that \textbf{grounding is one of the key elements to bring human-like language understanding}. However, language grounding is a very broad area that covers a great diversity of techniques, modalities and concepts. In this project, we will focus on spatial reasoning, that is, \textbf{grounding LMs with spatial concepts}. We choose spatial reasoning because it is one of the most fundamental capabilities for both humans and LMs. Such relations are crucial to how humans organize the mental space and make sense of the physical world, and therefore fundamental for a grounded theory of semantics \cite{levinson2003space}. However, spatial reasoning has been found to be particularly challenging (much more challenging than capturing properties of individual entities) for current models \cite{akula2020words}.

Vision Language Models (VLM), which are trained jointly on text and image, have been proposed as a general solution to the lack of grounding in language models \cite{lu2019vilbert, tan2020lxmert, ramesh2022hierarchical, saharia2022photorealistic}. VLMs have been used in tasks that require grounding spatial concepts, such as VQA \cite{antol2015vqa} or NLVR2 \cite{suhr2018corpus}, but recent work has shown that \textbf{VLMs struggle to ground spatial concepts properly} \cite{liu2022things}. Large generative VLMs trained on massive amounts of data like DALLE-2 \cite{ramesh2022hierarchical} or IMAGEN \cite{saharia2022photorealistic} are known to possess visual-reasoning skills \cite{cho2022dall}, but they are not publicly available and only accessible to large companies.

There are several \textbf{works that try to ground language models to spatial relations}. For example, \cite{bagherinezhad2016elephants, elazar2019large} focus on the acquired commonsense knowledge of models about object scales, e.g. do they know that a person is bigger than an ant? However, they ask about generic object scale relations, without providing any context. Some other authors \cite{collell2018acquiring, elu2021inferring} work on implicit and explicit spatial relations of objects, given some descriptive texts. The proposed benchmark datasets are designed for object bounding box generation.

Multimodal training datasets with images and corresponding textual descriptions that include explicitly spatial relations tend to be small. A very recent work proposes a method called Pseudo-Q to \textbf{automatically create synthetic datasets that can be used to train visually grounded models} \cite{jiang2022pseudo}. Their method consists of leveraging an off-the-shelf object detector to identify visual objects from unlabeled images, and then creating language queries for these objects that are obtained in an unsupervised fashion with a pseudo-query generation module. We propose to follow a similar approach, and create synthetic datasets that are specially tailored to acquire spatial relations.

With the objective of \textbf{evaluating spatial relations}, a recent work provides new unified datasets \cite{liu2022things}. As the objective of such work is to evaluate whether VLMs learn more spatial commonsense than LMs, the datasets are purely textual, so they do not provide any means to ground spatial concepts (they assume the grounding occurs in a previous training process). Interestingly, authors find that VLMs, and more concretely text-to-image systems, perform much better than text-only LMs. 

CLEVR was one of the pioneering works on testing compositional language and elementary visual reasoning \cite{johnson2017clevr}. However, it presents two major drawbacks: i) questions not only cover spatial grounding but some other concepts such as compositional language and attribute identification, and ii) spatial relations are limited to four, i.e. left, right, behind and in front. In a similar fashion, SpartQA provides a synthetic question-answering dataset that is specially focused on spatial reasoning capabilities. However, it contains only text and no images, and therefore it does not provide any means to ground spatial concepts. 

The Winoground dataset \cite{thrush2022winoground} is focused on \textbf{evaluating visio-linguistic compositional reasoning} in VLMs. Each instance in the dataset is composed of two images and two captions, but crucially, both captions contain a completely identical set of words, only in a different order. The task is then to match them correctly, which requires the systems to properly deal with composition in natural language.

Another very recent dataset named Visual Spatial Reasoning (VSR) \cite{liu2022visual}, whose objective is to test spatial grounding capabilities by covering 65 different spatial relations over natural images collected from COCO \cite{lin2014microsoft}. Given an image, VSR provides a caption which describes a spatial relation between two of the objects that appear in the image. That relation can be real or fake, and that is precisely what the model has to infer, i.e. whether the caption is correct with respect to the given image. Another advantage of this dataset is that it is annotated by humans. Given its features, \textbf{we believe VSR is a good candidate to evaluate spatial grounding in LMs}.

\section{Objectives}

Despite the impressive performance of pretrained vision and language models (VLMs) on a wide variety of multimodal tasks, they remain poorly understood. One important question is to what extent such models are able to conduct unimodal and multimodal compositional reasoning and spatial reasoning. For example, the visual differences between images depicting "a person sits and a dog stands" and "a person stands and a dog sits" are clamorously obvious for humans, but still not clear for current state-of-the-art VLMs. To perform well on tasks where compositional and spatial reasoning is required, the models do not only need a proper encoding of text and images, but also to be able to \textbf{ground meaning across the two modalities} (spatial grounding).

Thus the main objective of the project is to \textbf{learn language models for spatial reasoning via the grounding of LMs with spatial concepts and relations}. One of the main goals of the project is to investigate ways to acquire grounded representation for spatial reasoning. In that sense, we will define suitable ways to incorporate spatial information into pre-trained vision and language models. Towards this goal, this project will focus on using the latest advances in deep-learning techniques, pre-trained LMs for effective zero and few-shot transfer learning.

We have defined the following specific objectives in the scope of spatial reasoning:

\begin{enumerate}
    \item \textbf{Investigate the use of synthetic datasets to overcome the lack of annotated datasets for spatial grounding}. As to avoid the scarcity of multimodal datasets that explicitly describe spatial relations, we propose to automatically construct synthetic datasets on spatial relations and use them to train existing language models in a self-supervised way, with the final aid of obtaining spatially grounded language models. In particular, we propose two alternatives to produce the synthetic datasets:
    \begin{enumerate}
        \item \textbf{Explicit verbalization} of spatial relations in images. Given an image in an existing dataset, we propose to use an object detector to identify the entities in the images, as well as hand-designed verbalization templates to automatically generate textual descriptions of the spatial relations among them.
        \item \textbf{Using large generative VLMs}, which are known to obey spatial relations as described in the text, to obtain realistic images with entities that are arranged following certain spatial relations.
    \end{enumerate}
    \item \textbf{Investigate the use of multi-tasking and multi-sourcing to improve generalization properties}. In a multi-task training paradigm, the model is forced to learn more than one task simultaneously, therefore improving its generalization capabilities. We will investigate multi-task settings to combine the verbalized dataset, the images produced by the generative VLMs, as well as traditional training data to obtain spatial-aware language models.
    \item \textbf{Improve zero-shot and few-shot generalization of VLM models} to obtain effective models in small data regimes of the spatial reasoning domain without the necessity of explicitly annotating big quantities of spatial relations.
    \item \textbf{Improve the state of the art in spatial reasoning}. Improve the state of the art in spatial reasoning. The final goal is to apply the findings learnt from previous objectives to improve the state-of-the-art in multiple datasets. We plan to evaluate our models at least on two vision and language datasets. The first one is the Winoground dataset \cite{thrush2022winoground}, which presents a novel task for evaluating the ability of vision and language models to conduct visio-linguistic compositional reasoning. The second one is the VSR benchmark \cite{liu2022visual} for investigating VLMs capabilities in recognising 65 types of spatial relationships in natural text-image pairs.
\end{enumerate}

\section{Research Questions}

Research Tasks (RT) and Questions (RQ) are based on the objectives from the previous section.

\textbf{RT0. Prepare the research scenario}. The initial task is related to \textbf{gathering corpora, exploring different datasets}, \textbf{Language Models} (LM) and \textbf{building a baseline prototype}. We have already identified some important datasets on spatial reasoning but we will check if there is any new appropriate dataset to evaluate our models. At the same time, we will examine and reimplement (if needed) state-of-the-art systems in order to further understand the task to be solved. This leads us to the following research questions: \textbf{RQ0.A) Are the available datasets appropriate to evaluate the spatial abilities of current LMs?} \textbf{RQ0.B) Which is the best pre-trained LMs for spatial reasoning?} We will conduct a quantitative and qualitative analysis of the existing text-only LMs and vision-language LMs in order to 1) measure the appropriateness of probing evaluations of the datasets and 2) explain the limitations of different types of pre-trained LMs.

\textbf{RT2: Perform synthetic data generation using generative models to learn spatial grounding}. We will focus on using large generative VLMs to construct high quality synthetic images that depict a fixed set of spatial relations. In that sense, we want to answer the following research questions. \textbf{RQ2.A) Which is the right way to make explicit the implicit information encoded in generative VLMs?} \textbf{RQ2.B) Can we improve the state-of-the-art of vision and language models in tasks that require spatial reasoning?}

\textbf{RT3: Perform multi-task and multi-source learning in few-shot settings}. In this task we will focus on finding ways of applying multi-task learning using multiple sources of information in order to force LMs to ground spatial relations into text without the necessity of explicitly annotating big quantities of spatial relations. This leads us to the following research questions: \textbf{RQ3.A) What kind of tasks and information sources are relevant to learn spatial information effectively?} \textbf{RQ3.B) What is the best way to combine the task in a multi-task setting?} \textbf{RQ3.C) Can we effectively minimize annotated data to obtain state-of-the-art results in tasks that require spatial reasoning?}