This chapter explains the methods we used for evaluation in each dataset. This includes model configurations and other settings.

\section{Winoground}

We introduce baseline models and our models in \Cref{sec:wino_models} and then provide an overview of all the models.

\subsection{Models} \label{sec:wino_models}

\paragraph{Baselines.}
Winoground authors \cite{thrush2022winoground} evaluate various configurations of the following multimodal transformers: CLIP \cite{radford2021clip}, FLAVA \cite{singh2022flava}, LXMERT \cite{tan2020lxmert}, UniT \cite{hu2021unit}, UNITER \cite{chen2020uniter}, VILLA \cite{gan2020villa}, VinVL \cite{zhang2021vinvl}, ViLT \cite{kim2021vilt}, VisualBERT \cite{li2019visualbert} and ViLBERT \cite{lu2019vilbert}. They also evaluate several configurations of two types of RNN-based models: VSE++ \cite{faghri2018vse} and VSRN \cite{li2019vsrn}.

\paragraph{Ours.}
We evaluate various configurations of the following multimodal transformers: OFA \cite{wang2022unifying}, BLIP \cite{li2022blip}, CLIP \cite{radford2021clip}, FLAVA \cite{singh2022flava} and ViLT \cite{kim2021vilt}. OFA and BLIP were not included in the baseline evaluation. The other models were already included but we test more configurations. For example, we test models that are finetuned on Flickr30k, COCO, NLVR2 and VSR. We try different model sizes when they are available.

\paragraph{Overview.}
We provide a high-level overview of the differences between the models in \cref{tab:model-types} which includes pretraining datasets, architecture, and attention mechanisms between the modalities. We omit datasets that were only used to train backbones. We exclude the language embedding from this table as every model uses a pretrained BERT tokenizer, except CLIP, VSE++, and VSRN. The pretraining datasets include COCO \cite{lin2014microsoft}, Visual Genome (VG) \cite{krishna2016visual}, Conceptual Captions (CC) \cite{sharma2018conceptual}, SBU Captions \cite{ordonez2011im2text}, Flickr30k \cite{young2014image}, VQA 2.0 \cite{goyal2017making}, VCR \cite{zellers2019recognition}, NLVR2 \cite{suhr2017corpus}, SNLI-VE \cite{xie2018visual}, QNLI \cite{rajpurkar2016squad}, MLNI-mm \cite{williams2017broad}, QQP \cite{QQPDataset}, Localized Narratives (LN) \cite{pont-tuset2020localized-narratives}, Wikipedia Image Text (WIT) \cite{srinivasan2021wit}, Conceptual Captions 12M (CC 12M) \cite{changpinyo2021conceptual12m}, Red Caps (RC) \cite{desai2021redcaps}, YFCC100M \cite{thomee2016yfcc100m}, SST-2 \cite{Socher2013RecursiveDM}, and LAION \cite{schuhmann2021laion}. CLIP uses their own dataset for pretraining.

\begin{table}[ht]
    \centering
    \small
    \begin{adjustbox}{max width=\textwidth}
    \begin{tabular}{l|lr|l|l}
    \toprule
    Model & Datasets & \# Images, Captions & Architecture & Attention \\\midrule
    VinVL \cite{zhang2021vinvl}  & VQA, GQA, VG-QA, COCO, Flickr30k, CC, SBU & 1.89, 4.87 & single-stream & merged \\
    UNITER \cite{chen2020uniter}  & COCO, VG, CC, SBU & 4.20, 9.58 & single-stream  &  merged \\
    ViLLA \cite{gan2020villa} & COCO, VG, CC, SBU  & 4.20, 9.58 & single-stream  &  merged \\
    VisualBERT \cite{li2019visualbert}& COCO, NVLR2 & 0.30, 0.52  & single-stream  & merged \\
    ViLT \cite{kim2021vilt}  & COCO, VG, SBU, CC & 4.10, 9.85 & single-stream  & merged \\
    LXMERT \cite{tan2020lxmert}  & COCO, VG & 0.18, 9.18 & dual-stream & modality-specific, co-attn, merged \\
    ViLBERT \cite{lu2019vilbert}  & CC & 3.30, 3.30 & dual-stream  & modality-specific, co-attn, merged \\
    UniT \cite{hu2021unit} & COCO detect., VG detect., VQAv2, SNLI-VE QNLI, MNLI-mm, QQP, SST-2 & 0.69, 1.91 & dual-stream & modality-specific, merged\\
    FLAVA $_{ITM}$ \cite{singh2022flava}  & COCO, SBU, LN, CC, VG, WIT, CC 12M, RC, YFCC100M & 70.00, 70.00 & dual-stream & modality-specific, merged \\
    FLAVA $_{Contrastive}$ \cite{singh2022flava}  & COCO, SBU, LN, CC, VG, WIT, CC 12M, RC, YFCC100M & 70.00, 70.00 & dual-stream & modality-specific \\
    CLIP \cite{radford2021clip}  & $-$ & 400.00, 400.00 & dual-stream & modality-specific \\
    \midrule
    OFA \cite{wang2022unifying} &  CC 12M, CC 3M, SBU, COCO, VG-Cap  &  20.00, 20.00 &   single-stream & modality-specific, merged \\
    BLIP$_{ITM}$ 14M \cite{li2022blip} &  COCO, VG, SBU, CC, CC 12M  &  14.00, 15.00 &   dual-stream & modality-specific, merged \\
    BLIP$_{ITC}$ 14M \cite{li2022blip} &  COCO, VG, SBU, CC, CC 12M & 14.00, 15.00 &   dual-stream &         modality-specific \\
    BLIP$_{ITM}$ 129M \cite{li2022blip} & COCO, VG, SBU, CC, CC 12M, LAION & 129.00,   130.00 &   dual-stream & modality-specific, merged \\
    BLIP$_{ITC}$ 129M \cite{li2022blip} & COCO, VG, SBU, CC, CC 12M, LAION & 129.00,   130.00 &   dual-stream &         modality-specific \\
    \bottomrule
    \end{tabular}
    \end{adjustbox}
    \caption{A high-level overview of the differences between the models we evaluate by the pretraining datasets, architecture, and attention mechanisms between the modalities.}
    \label{tab:model-types}
\end{table}

\subsection{Text-to-Image Generation}

We used Stable Diffusion to generate images 9 images for each Winoground caption. This results in a total of $800*9=7200$ images. This way we can test the compositional reasoning capabilities of a SOTA diffusion model.

We used Label Studio to annotate images generated by Stable Diffusion. As annotating all the images would take a very long time, we choose to annotate a subset of examples, and only one image per caption.

In each annotation there are two captions from Winoground and two images generated with Stable Diffusion. Each image is created from one caption but the order of the images is random. The annotators have to choose which text corresponds to each image: the first caption, the second caption, both or none. An screenshot of the annotation interface can be seen in...

There were 5 annotators in total and each one annotated 50 examples, for a total of 250 annotated examples. There are 400 examples in total, so we decided that it is a big enough subset.

The general conclusion is that Stable Diffusion is not good at this task. Most of the images do not match any of the captions. There are a few images that match both captions. The remaining images match one caption or the other, but there are many that match the incorrect caption. If we take into account image pairs, there are only a few correct ones.

\subsection{Image Captioning}

We used OFA \cite{wang2022unifying} and BLIP \cite{li2022blip} models of different sizes to generate captions for all Winoground images. We chose these models because they are SOTA in image captioning and we also use them in other evaluations. Our intention was to compare them with the real captions. We calculated BLEU scores for all models and we found out that they are very low. This indicates that the captions generated by these models are very far from the real captions. 

One reason for this could be that the real Winoground captions are not typical captions. They are hand-crafted so that they contain the same words in a different order, and that conditions the captions. Another reason could be that these models are not good at describing these types of images that require compositional reasoning. As these models are not very good at matching Winoground images with captions.

Analysing the captions manually would be necessary to know how good they really are. If these captions are good enough they could also be used to improve the results of the models by incorporating them in the evaluation process. They could provide extra information about the images to the models, that is not included in the original captions.

\subsection{Image Retrieval}

We used CLIP retrieval\footnote{\url{https://github.com/rom1504/clip-retrieval}} to retrieve images from LAION-5B dataset. We used Winoground captions and images to get similar images. For each caption and image, we compute its embeddings using CLIP ViT-L-14. Then the system uses a KNN algorithm to retrieve images that have similar embeddings. We can also compute the mean of caption and image embeddings to retrieve images that match both the image and the caption.

The system also has an aesthetic score that can be used to retrieve better looking images. It can also remove duplicate images and images that contain unsafe content and violence.

This system can be used to increase the size of our dataset. We can retrieve many similar images. The number of retrieved images and the similarity score can also be used as a measure of how common an image is. If there are very few similar images in the dataset, that means that the image is uncommon.

\section{Visual Spatial Reasoning}

We first introduce how the dataset is split for experiments in \Cref{sec:vsr_splits}, and then baseline and our models in \Cref{sec:vsr_models}.

\subsection{Dataset Splits}\label{sec:vsr_splits}

The VSR dataset has two types of splits \cite{liu2022visual}, random and zero-shot. The statistics of the two splits are shown in \cref{tab:data_splits}.

\begin{table}[ht]
\small
\centering
\begin{tabular}{lllll}
\toprule
 split & train & dev & test & total   \\
\midrule
\textit{random} & 7,083 & 1,012 & 2,024 & 10,119 \\
\textit{zero-shot} & 5,440 & 259 & 731  & 6,430\\
\bottomrule
\end{tabular}
\caption{Data statistics of the \textit{random} and \textit{zero-shot} splits. }
\label{tab:data_splits}
\end{table}

\paragraph{Random split.}
We split the dataset randomly into train/dev/test with the ratio of 70\%/10\%/20\%. All the validated data points are used in this split.

\paragraph{Zero-shot split.}
We create another concept zero-shot split where train/dev/test have no overlapping concepts. That is, each concept can only appear in one of the sets.
This is done by randomly grouping concepts into three sets with the ratio of 50\%/20\%/30\%.
This is a more challenging setup because the model has to learn concepts and relations in a compositional way instead of remembering the co-occurrence of the two.
Moreover, having less training data is a disadvantage for the models, since not all the data can be used in this setting.

\subsection{Models}\label{sec:vsr_models}

\paragraph{Baselines.} VSR authors \cite{liu2022visual} test three popular VLMs: VisualBERT \cite{li2019visualbert}, 
LXMERT \cite{tan2020lxmert}, and
ViLT \cite{kim2021vilt}. All three models are stacked Transformers \cite{vaswani2017attention} that take image and text pairs as input. The difference mainly lies in how or whether they encode position information of objects. Checkpoints are saved every 100 iterations and the best checkpoint on the dev set is used for testing. All models are run three times using three random seeds.

\paragraph{Ours.} We first test the same baseline models. We also evaluate a ViLT model that has only been finetuned on NLVR2. We evaluate BLIP \cite{li2022blip} trained on VSR and NLVR2.
